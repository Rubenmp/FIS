\documentclass[11pt,spanish]{article} % Idioma
\usepackage{babel}
\usepackage[T1]{fontenc}
\usepackage{textcomp, verbatim} % \begin{comment}
\usepackage[utf8]{inputenc} % Permite acentos

\usepackage{wrapfig} % Imagenes %\graphicspath{ {./imagenes/} }
\usepackage[left=2.75cm,top=2.5cm,right=2cm,bottom=2.5cm]{geometry} % Márgenes
\usepackage{amssymb, amsmath, amscd, amsfonts, amsthm, mathrsfs } % Símbolos matemáticos
\usepackage{cancel} % Cancelar expresiones
\usepackage{multirow, multicol, tabularx, booktabs, longtable} % Tablas
\usepackage{fancyhdr, fncychap} % Encabezados
\usepackage{algpseudocode, algorithmicx, algorithm} % Pseudo-código	
\usepackage{bbding} % Símbolos
\usepackage{enumitem} % Enumerados a), b), c)... usando \begin{enumerate}[label=\alph*)]
\usepackage{graphicx, xcolor, color, pstricks} % Gráficos --TikZ-- 
% http://www.texample.net/tikz/examples/
\usepackage[hidelinks]{hyperref}  % Enlaces
\usepackage{verbatim} % Comentarios largos \begin{comment}
\usepackage{rotating} % \begin{rotate}{30}
\usepackage[all]{xy} % Diagramas
\usepackage{xparse} % Entornos
\usepackage{listings}

\definecolor{codegreen}{rgb}{0,0.6,0}
\definecolor{codegray}{rgb}{0.5,0.5,0.5}
\definecolor{codepurple}{rgb}{0.58,0,0.82}
\definecolor{backcolour}{rgb}{0.95,0.95,0.92}

\lstdefinestyle{mystyle}{
	backgroundcolor=\color{backcolour},   
	commentstyle=\color{codegreen},
	keywordstyle=\color{magenta},
	numberstyle=\tiny\color{codegray},
	stringstyle=\color{codepurple},
	basicstyle=\footnotesize,
	breakatwhitespace=false,         
	breaklines=true,                 
	captionpos=b,                    
	keepspaces=true,                 
	numbers=left,                    
	numbersep=5pt,                  
	showspaces=false,                
	showstringspaces=false,
	showtabs=false,                  
	tabsize=2
}
\lstset{style=mystyle}


% Comandos
\newcommand{\docdate}{}
\newcommand{\subject}{}
\newcommand{\docauthor}{Rubén Morales Pérez}
\newcommand{\docemail}{srmorales@correo.ugr.es}

\newcommand{\N}{\mathbb{N}}
\newcommand{\Q}{\mathbb{Q}}
\newcommand{\C}{\mathbb{C}}
\newcommand{\R}{\mathbb{R}}
\newcommand{\Z}{\mathbb{Z}}


\linespread{1.1}                  % Espacio entre líneas.
\setlength\parindent{0pt}         % Indentación para párrafo.

\title{Fundamentos de la Ingeniería del Software \\ Prácticas}
\author{Alicia Rodríguez Gómez \\ Francisco Javier Morales Piqueras\\ Rubén Morales Pérez\\ Samia Mikou}
\date{ }





% % % % % % % % % % % % % % % % % % % % % % % % % % % % % % % % %
%					 Inicio del documento
% % % % % % % % % % % % % % % % % % % % % % % % % % % % % % % % %
\begin{document}

\maketitle
\tableofcontents % Generando el indice
\newpage
\setlength\parindent{0pt} % Quitamos la sangría


\section{Introducción}
Se debe desarrollar un software que permita alquilar (total o parcialmente) y compartir coches y/o viviendas. 
La idea principal es crear un sistema de economía colaborativa en el que los usuarios sean a la vez quienes prestan y quienes reciben un servicio. 
Una vez registrado un usuario podrá publicar y consultar anuncios, e interactuar con otros usuarios.

Los usuarios podrán dejar opiniones sobre los servicios recibidos y se establecerá un estándar de calidad que permitirá, antes de solicitar un servicio, tener una idea de la relación calidad/precio del mismo.

El sistema mantendrá la consistencia de dichos servicios, de forma que en todo momento sepamos las operaciones que se están llevando a cabo.
Dispondrá de un sistema de control ante errores para no lastrar la experiencia del usuario, además, el historial de operaciones del sistema nos permitirá mejorar la calidad de los servicios que en él se ofrecen usando un modelo que cambiará dinámicamente.

Hay características comunes a ambos tipos de alquileres, como las modalidades de pago.
El dominio del problema en el caso de alquiler de vehículos incluiría los puntos de recogida y destino, las fechas, número de plazas, si el usuario es el conductor o no.
Para la vivienda el dominio incluiría los periodos de alquiler, épocas del año, dirección, tipo de alquiler, tipo de vivienda, datos adicionales como el wi-fi, luz o agua incluidas, etc.


\section{Estructura de requisitos}
\subsection{Requisitos de información}
\begin{itemize}
	\item \textbf{RI-1} Usuarios
	
	Información sobre los usuarios del sistema.
	
	Identificador de usuario, DNI, calificación, descripción (opcional), teléfono, e-mail (opcional), rango, mensajes, cuenta bancaria (opcional), foto (opcional), fecha de alta como socio, saldo actual, alquileres vigentes, opiniones dejadas y recibidas.
	
	\item \textbf{RI-2} Vehículos
	
	Vehículos que han pasado alguna vez por nuestro sistema.
	
	Identificador del vehículo, matrícula, calificación, modelo, color, número de asientos (contando el conductor), concesionario, identificador del dueño
	
	\item \textbf{RI-3} Viviendas
	
	Viviendas que han pasado alguna vez por nuestro sistema.
	
	Identificador de la vivienda, identificador del dueño, calificación, ubicación, metros cuadrados, número de habitaciones, número de baños, datos adicionales (fotos, información sobre calefacción, garaje, ascensor, etc)
	
	\item \textbf{RI-4} Historial
	
	Historial de alquileres de cada usuario del sistema.
	
	Identificador de usuario, lista de alquileres (identificadores)
	% Pagos que tiene pendientes	
		
	\item \textbf{RI-5} Vehículos para alquilar
	
	Descripción de los vehículos disponibles para que otros usuarios los alquilen.
	
	Identificador del alquiler, identificador del vehículo, tipo de alquiler (total o parcial), datos adicionales.
	\begin{itemize}
		\item \textbf{RI-5.1} Alquiler total
		
		Alquiler del vehículo completo por un periodo de tiempo.
		
		Fecha inicio del alquiler, fecha fin del alquiler, precio por día alquilado, posibilidad de alquileres intermedios.
		
		\item \textbf{RI-5.2} Alquiler plazas para viaje
		
		En este caso se alquilarían plazas sueltas para un viaje concreto.
		
		Fecha de inicio del viaje, fecha estimada de llegada (derivado del resto), lugar inicial y final del viaje, posibles paradas intermedias (para recoger otros pasajeros), número de plazas ofertadas, precio por plaza, tamaño máximo del equipaje (pequeño, mediano o grande) y usuarios que ya han reservado alguna plaza.
	\end{itemize}

	\item \textbf{RI-6} Viviendas para alquilar
	
	Viviendas actualmente disponibles para alquiler total o parcial.
	
	Identificador de alquiler, identificador de la vivienda, tipo de alquiler (total o por habitaciones). 
		
	\begin{itemize}
		\item \textbf{RI-6.1} Alquiler total
		
		Alquiler de la vivienda completa por un periodo de tiempo
		
		Fecha inicio del alquiler, fecha fin del alquiler, precio por día alquilado, posibilidad de alquileres intermedios.
		
		\item \textbf{RI-6.2} Alquiler por habitaciones
		
		Alquiler de habitaciones sueltas.
		
		Fecha inicio del alquiler, fecha fin del alquiler, precio por día alquilado, posibilidad de alquileres intermedios, número de habitaciones para alquilar.
	\end{itemize}

% Politica de privacidad, normas, proveedores de seguros	
\end{itemize}

\subsection{Requisitos funcionales}
\begin{itemize}
	\item Gestión de usuarios
	
	\begin{itemize}
		\item Alta de usuario
		\item Consulta de usuario
		\begin{itemize}
			\item Datos personales
			\item Opiniones recibidas
			\item Alquileres efectuados
		\end{itemize}
		\item Modificar socio
		\begin{itemize}
			\item Modificar datos personales
			\begin{itemize}
				\item Verificar e-mail
				\item Verificar teléfono
			\end{itemize}
			\item Modificar status
			\item Añadir opinión
			\item Añadir alquiler efectuado
		\end{itemize}
		\item Gestión de pagos
		\begin{itemize}
			\item Consultar saldo
			\item Modificar saldo
			\item Pago a cuenta
		\end{itemize}
		\item Denunciar usuario
		\item Eliminar usuario
	\end{itemize}
	
	\item Gestión de vehículos
	\begin{itemize}
		\item Añadir vehículo
		\item Consultar vehículo
		\item Modificar vehículo
		\item Eliminar vehículo
	\end{itemize}
	\item Gestión de viviendas
	\begin{itemize}
		\item Añadir vivienda
		\item Consultar vivienda
		\item Modificar vivienda
		\item Eliminar vivienda
	\end{itemize}
	
	\item Gestión de alquileres
	\begin{itemize}
		\item Añadir alquiler
		\item Consulta alquiler
		\item Modificar alquiler
		\begin{itemize}
			\item Modificar información alquiler
			\item Solicitar aceptación de alquiler (total o parcial)
		\end{itemize}
		\item Gestión de mensajes
		\begin{itemize}
			\item Enviar mensaje
			\item Validar mensaje
		\end{itemize}
		\item Dejar una opinión
		\item Gestión de incidencias
		\begin{itemize}
			\item Notificar una incidencia 
			\item Consultar incidencias
			\item Consultar seguro
			\item Consultar servicios legales		
		\end{itemize}
		\item Archivar alquiler
		\item Eliminar alquiler
	\end{itemize}
	
	\item Consultar información legal
	\item Consultar normativa interna
	\item Contacto
	
	
\end{itemize}

\subsection{Requisitos no funcionales}
\begin{itemize}
	\item Mantener coherencia de los datos
	\item Controlar privilegios de usuarios y administradores
	
\end{itemize}

\section{Glosario de términos}
\begin{itemize}
	\item Alquiler intermedio: Alquiler que pueda empezar y/o terminar en otro momento distinto del especificado en el rango inicial
	\item Calificación: Puntuación numérica entre $0$ y $5$
	\item Economía colaborativa
	\item Historial
	\item Incidencia (retraso, información falsa, comportamiento inadecuado, etc.)
	\item Status: Nivel de experiencia de un usuario
	\item Ubicación: Coordenadas del sitio al que estamos referenciando.
	\item Validar	
\end{itemize}


%%%%%%%%%%%%%%%%%%%%%%%%%%%%%%%%%%%%%%%%%%%%%%%%%%%%%%%%%%%%%%%%%%%%%%%%%%%%%%%%%%

% % % % % % % % % % % % % % % % % % % % % % % % % % % % % % % % %
%					 Bibliografía
% % % % % % % % % % % % % % % % % % % % % % % % % % % % % % % % %

% Para hacer citas: \cite{<referencia>}
% donde <referencia> es lo primero que aparece en el fichero .bib:
% @tipo{<referencia>, ...} 

% Si no se utiliza una cita hay que especificarlo con \nocite{<referencia>}


% El orden de referencias es determinado por el orden de aparición de \cite y \nocite
\bibliographystyle{unsrt}  

\bibliography{./Bibliografia/preambulo,./Bibliografia/referencias}  


\end{document}
