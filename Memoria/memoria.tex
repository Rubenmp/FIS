\documentclass[11pt,spanish]{article} % Idioma
\usepackage{babel}
\usepackage[T1]{fontenc}
\usepackage{textcomp, verbatim} % \begin{comment}
\usepackage[utf8]{inputenc} % Permite acentos

\usepackage{wrapfig} % Imagenes %\graphicspath{ {./imagenes/} }
\usepackage[left=2.75cm,top=2.5cm,right=2cm,bottom=2.5cm]{geometry} % Márgenes
\usepackage{amssymb, amsmath, amscd, amsfonts, amsthm, mathrsfs } % Símbolos matemáticos
\usepackage{cancel} % Cancelar expresiones
\usepackage{multirow, multicol, tabularx, booktabs, longtable} % Tablas
\usepackage{fancyhdr, fncychap} % Encabezados
\usepackage{algpseudocode, algorithmicx, algorithm} % Pseudo-código	
\usepackage{bbding} % Símbolos
\usepackage{enumitem} % Enumerados a), b), c)... usando \begin{enumerate}[label=\alph*)]
\usepackage{graphicx, xcolor, color, pstricks} % Gráficos --TikZ-- 
% http://www.texample.net/tikz/examples/
\usepackage[hidelinks]{hyperref}  % Enlaces
\usepackage{verbatim} % Comentarios largos \begin{comment}
\usepackage{rotating} % \begin{rotate}{30}
\usepackage[all]{xy} % Diagramas
\usepackage{xparse} % Entornos
\usepackage{listings}

\definecolor{codegreen}{rgb}{0,0.6,0}
\definecolor{codegray}{rgb}{0.5,0.5,0.5}
\definecolor{codepurple}{rgb}{0.58,0,0.82}
\definecolor{backcolour}{rgb}{0.95,0.95,0.92}

\lstdefinestyle{mystyle}{
	backgroundcolor=\color{backcolour},   
	commentstyle=\color{codegreen},
	keywordstyle=\color{magenta},
	numberstyle=\tiny\color{codegray},
	stringstyle=\color{codepurple},
	basicstyle=\footnotesize,
	breakatwhitespace=false,         
	breaklines=true,                 
	captionpos=b,                    
	keepspaces=true,                 
	numbers=left,                    
	numbersep=5pt,                  
	showspaces=false,                
	showstringspaces=false,
	showtabs=false,                  
	tabsize=2
}
\lstset{style=mystyle}


% Comandos
\newcommand{\docdate}{}
\newcommand{\subject}{}
\newcommand{\docauthor}{Rubén Morales Pérez}
\newcommand{\docemail}{srmorales@correo.ugr.es}

\newcommand{\N}{\mathbb{N}}
\newcommand{\Q}{\mathbb{Q}}
\newcommand{\C}{\mathbb{C}}
\newcommand{\R}{\mathbb{R}}
\newcommand{\Z}{\mathbb{Z}}


\linespread{1.1}                  % Espacio entre líneas.
\setlength\parindent{0pt}         % Indentación para párrafo.

\title{Fundamentos de la Ingeniería del Software \\ Prácticas}
\author{Alicia Rodríguez Gómez \\ Francisco Javier Morales Piqueras\\ Rubén Morales Pérez\\ Samia Mikou}
\date{ }





% % % % % % % % % % % % % % % % % % % % % % % % % % % % % % % % %
%					 Inicio del documento
% % % % % % % % % % % % % % % % % % % % % % % % % % % % % % % % %
\begin{document}

\maketitle
\tableofcontents % Generando el indice
\newpage
\setlength\parindent{0pt} % Quitamos la sangría


\section{Introducción}
\hspace{0.5cm}Se debe desarrollar un software que permita alquilar (total o parcialmente) y compartir coches y/o viviendas. 
La idea principal es crear un sistema de economía colaborativa en el que los usuarios sean a la vez quienes prestan y quienes reciben un servicio. 
Una vez registrado un usuario podrá publicar y consultar anuncios, e interactuar con otros usuarios.

\hspace{0.5cm}Los usuarios podrán dejar opiniones sobre los servicios recibidos y se establecerá un estándar de calidad que permitirá, antes de solicitar un servicio, tener una idea de la relación calidad/precio del mismo.

\hspace{0.5cm}El sistema mantendrá la consistencia de dichos servicios, de forma que en todo momento sepamos las operaciones que se están llevando a cabo.
Dispondrá de un sistema de control ante errores para no lastrar la experiencia del usuario, además, el historial de operaciones del sistema nos permitirá mejorar la calidad de los servicios que en él se ofrecen usando un modelo que cambiará dinámicamente.

\hspace{0.5cm}En lo referente al dominio del problema hay características comunes a ambos tipos de alquileres, como las modalidades de pago, las opiniones, el precio, disponibilidad de plazas o habitaciones.
En el caso de alquiler de vehículos incluiría los puntos de recogida y destino, las fechas, número de plazas, si el usuario es el conductor o no.
Para la vivienda el dominio incluiría los periodos de alquiler, dirección, tipo de alquiler, tipo de vivienda, datos adicionales como el Wi-Fi, luz o agua (en caso de estar o no incluidas en el precio), etc.

\section{Implicados}

% Please add the following required packages to your document preamble:
% \usepackage{graphicx}
\begin{table}[]
	\centering
	\label{my-label}
	\resizebox{\textwidth}{!}{%
		\begin{tabular}{llll}
			Nombre & Descripción & Responsabilidad & Criterios de éxito \\
			Administrador & Entidad que administra el software & Gestiona el buen uso de la aplicación y resuelve conflictos & Resolver efectivamente los conflictos y garantiza el buen funcionamiento del software. \\
			Arrendador & Entidad que posee la vivienda o vehículo a arrendar. & Publicar el anuncio de alquiler y la información de contacto. Cumplir los términos del contrato. & Ofrecer información veraz y completa, conseguir alquilar la vivienda o vehículo. \\
			Arrendatario & Entidad que paga al arrendador por el uso su vivienda o vehículo. & Hacer un uso responsable de la vivienda o vehículo. Cumplir con los términos del contrato. & Conseguir una vivienda o vehículo de alquiler y no dar problemas al arrendador. \\
			Conductor & Entidad que conduce el coche durante el trayecto contratado. & Cumplir con la normativa de tráfico, cumplir el horario previsto. & Conseguir pasajeros para su trayecto. \\
			Pasajero & Entidad que comparte el trayecto con el conductor. & Llegar a tiempo a la cita, respetar el vehículo y confirmar que ha realizado dicho trayecto. & Conseguir el trayecto deseado a buen precio. \\
			Vehículo & Vehículo ofertado para su uso en un periodo de tiempo o en un trayecto & No tiene responsabilidad. & No averiarse. \\
			Vivienda & Vivienda ofertada para su uso en un periodo de tiempo o para alquilar habitaciones. & No tiene responsabilidad. & Ser habitable.
		\end{tabular}%
	}
\end{table}




\section{Estructura de requisitos}
\subsection{Requisitos de información}
\begin{itemize}
	\item \textbf{RI-1} Usuarios
	
	Información sobre los usuarios del sistema.
	
	Identificador de usuario, DNI, calificación (derivada de otros usuarios), descripción (opcional), teléfono, e-mail (opcional), status, mensajes, cuenta bancaria (opcional), foto (opcional), fecha de alta como usuario, saldo virtual actual, alquileres vigentes y pasados, opiniones dejadas y recibidas.
	
	\item \textbf{RI-2} Vehículos
	
	Vehículos que han pasado alguna vez por nuestro sistema.
	
	Identificador del vehículo, matrícula, calificación, modelo, color, número de asientos (contando el conductor).
	También se incluye si se permiten mascotas, fumar, etc.
	
	\item \textbf{RI-3} Viviendas
	
	Viviendas que han pasado alguna vez por nuestro sistema.
	
	Identificador de la vivienda, identificador del dueño, calificación (derivada de otros usuarios), ubicación, inclusión de mobiliario, metros cuadrados, número de habitaciones, número de baños, tasas incluidas, datos adicionales (fotos, información sobre calefacción, garaje, ascensor, piscina, terraza, etc.).
	También se incluye si hay fianza.
	
	\item \textbf{RI-4} Historial
	
	Historial de alquileres de cada usuario del sistema.
	
	Identificador de usuario, lista de alquileres (identificadores)
	% Pagos que tiene pendientes	
		
	\item \textbf{RI-5} Vehículos para alquilar
	
	Descripción de los vehículos disponibles para que otros usuarios los alquilen.
	
	Identificador del alquiler, identificador del dueño, identificador del vehículo, tipo de alquiler (total o parcial), datos adicionales.
	\begin{itemize}
		\item \textbf{RI-5.1} Alquiler total
		
		Alquiler del vehículo completo por un periodo de tiempo.
		
		Fianza, fechas de inicio y fin del alquiler, precio, lugar de recogida y devolución del vehículo, posibilidad de alquileres intermedios.
		
		\item \textbf{RI-5.2} Alquiler plazas para viaje
		
		En este caso se alquilarían plazas sueltas para un viaje concreto.
		
		Fecha de inicio del viaje, fecha estimada de llegada (derivado del resto), lugar inicial y final del viaje, posibles paradas intermedias (para recoger otros pasajeros), número de plazas ofertadas y disponibles, precio por plaza, tamaño máximo del equipaje (pequeño, mediano o grande) y usuarios que ya han reservado alguna plaza en subtrayectos.
	\end{itemize}

	\item \textbf{RI-6} Viviendas para alquilar
	
	Viviendas actualmente disponibles para alquiler total o parcial.
	
	Identificador de alquiler, identificador de la vivienda, tipo de alquiler (total o por habitaciones). 
		
	\begin{itemize}
		\item \textbf{RI-6.1} Alquiler total
		
		Alquiler de la vivienda completa por un periodo de tiempo
		
		Fecha inicio del alquiler, fecha fin del alquiler, precio por día alquilado, posibilidad de alquileres intermedios.
		
		\item \textbf{RI-6.2} Alquiler por habitaciones
		
		Alquiler de habitaciones sueltas.
		
		Fecha inicio del alquiler, fecha fin del alquiler, precio por día alquilado, posibilidad de alquileres intermedios, número de habitaciones para alquilar.
	\end{itemize}

	\item \textbf{RI-7} Normas
	\item \textbf{RI-8} Proveedores de seguros
	\item \textbf{RI-9} Contacto con la empresa	
\end{itemize}

\subsection{Requisitos funcionales}
\begin{itemize}
	\item \textbf{RF-1} Gestión de usuarios
	\begin{itemize}
		\item \textbf{RF-1.1} Alta de usuario
		\item \textbf{RF-1.2} Consulta de usuario
		\begin{itemize}
			\item \textbf{RF-1.2.1} Datos personales
			\item \textbf{RF-1.2.2} Opiniones recibidas
			\item \textbf{RF-1.2.3} Alquileres efectuados
		\end{itemize}
		\item \textbf{RF-1.3} Modificar usuario
		\begin{itemize}
			\item \textbf{RF-1.3.1} Modificar datos personales
			\begin{itemize}
				\item \textbf{RF-1.3.1.1} Verificar e-mail
				\item \textbf{RF-1.3.1.2} Verificar teléfono
			\end{itemize}
			\item \textbf{RF-1.3.2} Añadir vehículo
			\item \textbf{RF-1.3.3} Añadir vivienda
			\item \textbf{RF-1.3.4} Modificar status
			\item \textbf{RF-1.3.5} Añadir alquiler efectuado
		\end{itemize}
		\item \textbf{RF-1.4} Gestión de pagos
		\begin{itemize}
			\item \textbf{RF-1.4.1} Consultar saldo
			\item \textbf{RF-1.4.2} Modificar saldo
			\item \textbf{RF-1.4.3} Pago a cuenta
		\end{itemize}
		\item \textbf{RF-1.5} Denunciar usuario
		\item \textbf{RF-1.6} Eliminar usuario
	\end{itemize}
	
	\item \textbf{RF-2} Gestión de alquileres
	\begin{itemize}
		\item \textbf{RF-2.1} Añadir alquiler
		\begin{itemize}
			\item \textbf{RF-2.1.1} Elegir vivienda
			\begin{itemize}
				\item \textbf{RF-2.1.1.1} Elegir habitación
			\end{itemize}
			\item \textbf{RF-2.1.2} Elegir vehículo
		\end{itemize}
		\item \textbf{RF-2.2} Consultar alquiler
		\item \textbf{RF-2.3} Modificar alquiler
		\begin{itemize}
			\item \textbf{RF-2.3.1} Modificar información alquiler
			\item \textbf{RF-2.3.2} Solicitar aceptación de alquiler (total o parcial)
			\item \textbf{RF-2.3.3} Confirmar realización de alquiler
		\end{itemize}
		\item \textbf{RF-2.4} Gestión de mensajes
		\begin{itemize}
			\item \textbf{RF-2.4.1} Enviar mensaje
			\item \textbf{RF-2.4.2} Validar mensaje
		\end{itemize}
		\item \textbf{RF-2.5} Dejar una calificación
		\item \textbf{RF-2.6} Gestión de incidencias
		\begin{itemize}
			\item \textbf{RF-2.6.1} Notificar una incidencia 
			\item \textbf{RF-2.6.2} Consultar incidencias
		\end{itemize}
		\item \textbf{RF-2.7} Retirar alquiler de lista de ofertados
	\end{itemize}	

	\item \textbf{RF-3} Gestión de administradores
	\begin{itemize}
		\item \textbf{RF-3.1} Consulta de incidencias
		\item \textbf{RF-3.2} Resolución de incidencias
		\item \textbf{RF-3.3} Gestión de privilegios
	\end{itemize}
\end{itemize}

\subsection{Requisitos no funcionales}
\begin{itemize}
	\item \textbf{RNF-1} Mantener coherencia de los datos
	\item \textbf{RNF-2} Privacidad de los datos personales de los usuarios
	\item \textbf{RNF-3} Interfaz intuitiva para alquilar	
\end{itemize}

\section{Glosario de términos}
\begin{itemize}
	\item Alquiler intermedio: Alquiler que pueda empezar y/o terminar en otro momento distinto del especificado en el rango inicial
	\item Calificación: Puntuación numérica entre $0$ y $5$ (incluidos), donde $0$ es la menor nota y $5$ la mayor.
	\item Economía colaborativa
	\item Historial: Lista de alquileres relativos a un usuario concreto (siendo dicho usuario el proveedor o el consumidor, indistintamente)
	\item Incidencia (retraso, información falsa, comportamiento inadecuado, etc.)
	\item Subtrayecto: Trayecto parcial del viaje, que empieza y/o acaba en paradas intermedias.
	\item Status: Nivel de experiencia de un usuario
	\item Ubicación: Coordenadas del sitio al que estamos referenciando.
	\item Verificar: Comprobar que el recurso pertenece al usuario indicado	
\end{itemize}


%%%%%%%%%%%%%%%%%%%%%%%%%%%%%%%%%%%%%%%%%%%%%%%%%%%%%%%%%%%%%%%%%%%%%%%%%%%%%%%%%%

% % % % % % % % % % % % % % % % % % % % % % % % % % % % % % % % %
%					 Bibliografía
% % % % % % % % % % % % % % % % % % % % % % % % % % % % % % % % %

% Para hacer citas: \cite{<referencia>}
% donde <referencia> es lo primero que aparece en el fichero .bib:
% @tipo{<referencia>, ...} 

% Si no se utiliza una cita hay que especificarlo con \nocite{<referencia>}


% El orden de referencias es determinado por el orden de aparición de \cite y \nocite
\bibliographystyle{unsrt}  

\bibliography{./Bibliografia/preambulo,./Bibliografia/referencias}  


\end{document}
