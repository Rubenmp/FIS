\documentclass[11pt,spanish]{article} % Idioma
\usepackage{babel}
\usepackage[T1]{fontenc}
\usepackage{textcomp, verbatim} % \begin{comment}
\usepackage[utf8]{inputenc} % Permite acentos

\usepackage{wrapfig} % Imagenes %\graphicspath{ {./imagenes/} }
\usepackage[left=2.75cm,top=2.5cm,right=2cm,bottom=2.5cm]{geometry} % Márgenes
\usepackage{amssymb, amsmath, amscd, amsfonts, amsthm, mathrsfs } % Símbolos matemáticos
\usepackage{cancel} % Cancelar expresiones
\usepackage{multirow, multicol, tabularx, booktabs, longtable} % Tablas
\usepackage{fancyhdr, fncychap} % Encabezados
\usepackage{algpseudocode, algorithmicx, algorithm} % Pseudo-código
\usepackage{bbding} % Símbolos
\usepackage{enumitem} % Enumerados a), b), c)... usando \begin{enumerate}[label=\alph*)]
\usepackage{graphicx, xcolor, color, pstricks} % Gráficos --TikZ--
% http://www.texample.net/tikz/examples/
\usepackage[hidelinks]{hyperref}  % Enlaces
\usepackage{verbatim} % Comentarios largos \begin{comment}
\usepackage{rotating} % \begin{rotate}{30}
\usepackage[all]{xy} % Diagramas
\usepackage{xparse} % Entornos
\usepackage{listings}







\definecolor{codegreen}{rgb}{0,0.6,0}
\definecolor{codegray}{rgb}{0.5,0.5,0.5}
\definecolor{codepurple}{rgb}{0.58,0,0.82}
\definecolor{backcolour}{rgb}{0.95,0.95,0.92}

\lstdefinestyle{mystyle}{
	backgroundcolor=\color{backcolour},
	commentstyle=\color{codegreen},
	keywordstyle=\color{magenta},
	numberstyle=\tiny\color{codegray},
	stringstyle=\color{codepurple},
	basicstyle=\footnotesize,
	breakatwhitespace=false,
	breaklines=true,
	captionpos=b,
	keepspaces=true,
	numbers=left,
	numbersep=5pt,
	showspaces=false,
	showstringspaces=false,
	showtabs=false,
	tabsize=2
}
\lstset{style=mystyle}


% Comandos
\newcommand{\docdate}{}
\newcommand{\subject}{}
\newcommand{\docauthor}{Rubén Morales Pérez}
\newcommand{\docemail}{srmorales@correo.ugr.es}

\newcommand{\N}{\mathbb{N}}
\newcommand{\Q}{\mathbb{Q}}
\newcommand{\C}{\mathbb{C}}
\newcommand{\R}{\mathbb{R}}
\newcommand{\Z}{\mathbb{Z}}

\usepackage{array} % for defining a new column type
\usepackage{varwidth} %for the varwidth minipage environment
\newcolumntype{M}{>{\begin{varwidth}{4cm}}l<{\end{varwidth}}} %M is for Maximal column

\linespread{1.1}                  % Espacio entre líneas.
\setlength\parindent{0pt}         % Indentación para párrafo.

\title{Fundamentos de la Ingeniería del Software \\ Prácticas}
\author{Alicia Rodríguez Gómez \\ Francisco Javier Morales Piqueras\\ Rubén Morales Pérez\\ Samia Mikou}
\date{ }





% % % % % % % % % % % % % % % % % % % % % % % % % % % % % % % % %
%					 Inicio del documento
% % % % % % % % % % % % % % % % % % % % % % % % % % % % % % % % %
\begin{document}

\maketitle
\tableofcontents % Generando el indice
\newpage
\setlength\parindent{0pt} % Quitamos la sangría


\section{Introducción}
\hspace{0.5cm}Este software permitirá alquilar (total o parcialmente) y compartir coches y/o viviendas.
La idea principal es crear un sistema de economía colaborativa en el que los usuarios (previo registro) sean a la vez quienes prestan y quienes reciben un servicio.
Los usuarios podrán dejar opiniones sobre los servicios recibidos y se establecerá un estándar de calidad que permitirá tener una idea de la relación calidad/precio del mismo.

\hspace{0.5cm}El sistema mantendrá la consistencia de dichos servicios, de forma que en todo momento sepamos las operaciones que se están llevando a cabo.
Dispondrá de un sistema de control ante errores para no lastrar la experiencia del usuario, además, el historial de operaciones del sistema nos permitirá mejorar la calidad de los servicios que en él se ofrecen usando un modelo que cambiará dinámicamente.

\hspace{0.5cm}En lo referente al dominio del problema hay características como las modalidades de pago, las opiniones, el precio, disponibilidad de plazas o habitaciones.
En el caso de alquiler de vehículos incluiría los puntos de recogida y destino, las fechas, número de plazas, si el usuario es el conductor o no.
Para la vivienda el dominio incluiría los periodos de alquiler, dirección, tipo de vivienda, datos adicionales como el Wi-Fi, luz o agua (en caso de estar o no incluidas en el precio), etc.

\section{Implicados}

\begin{table}[H]
	\centering
		\begin{tabular}{|p{0.9in}|p{1.5in}|p{2in}|p{2in}|}
			\hline
			Nombre & Descripción & Responsabilidad & Criterios de éxito \\ \hline
			
			Administrador & Entidad que administra el software & Gestiona el buen uso de la aplicación y resuelve conflictos & Resolver efectivamente los conflictos y garantiza el buen funcionamiento del software. \\ \hline
			
			Arrendador & Entidad que posee la vivienda o vehículo a arrendar. & Publicar el anuncio y la información de contacto. Cumplir los términos del contrato. & Ofrecer información veraz y completa, conseguir alquilar la vivienda o vehículo. \\ \hline
			
			Arrendatario & Entidad que paga al arrendador por el uso su vivienda o vehículo. & Hacer un uso responsable de la vivienda o vehículo. Cumplir con los términos del contrato. & Conseguir una vivienda o vehículo de alquiler y no dar problemas al arrendador. \\ \hline
			
			Conductor & Entidad que conduce el coche durante el trayecto contratado. & Cumplir con la normativa de tráfico, cumplir el horario previsto. & Conseguir pasajeros para su trayecto. \\ \hline
			
			Pasajero & Entidad que comparte el trayecto con el conductor. & Llegar a tiempo a la cita, respetar el vehículo y confirmar la realización del trayecto. & Conseguir el trayecto deseado a buen precio. \\ \hline
			
			Vehículo & Vehículo ofertado para su uso un periodo de tiempo o un trayecto & No tiene responsabilidad. & No averiarse. \\ \hline
			
			Vivienda & Vivienda ofertada para su uso en un periodo de tiempo o para alquilar habitaciones. & No tiene responsabilidad. & Ser habitable.\\ \hline
		\end{tabular}%

\end{table}




\section{Estructura de requisitos}
\subsection{Requisitos de información}
\begin{itemize}
	\item \textbf{RI-1} Usuarios

	Información sobre los usuarios del sistema.

	Identificador de usuario, DNI, calificación (derivada de otros usuarios), descripción (opcional), teléfono, e-mail (opcional), status, mensajes, cuenta bancaria (opcional), foto (opcional), fecha de alta como usuario, saldo virtual actual, alquileres vigentes y pasados, opiniones dejadas y recibidas.

	\item \textbf{RI-2} Vehículos

	Vehículos que han pasado alguna vez por nuestro sistema.

	Identificador del vehículo, matrícula, calificación, modelo, color, número de asientos (contando el conductor).
	También se incluye si se permiten mascotas, fumar, etc.

	\item \textbf{RI-3} Viviendas

	Viviendas que han pasado alguna vez por nuestro sistema.

	Identificador de la vivienda, identificador del dueño, calificación (derivada de otros usuarios), ubicación, inclusión de mobiliario, metros cuadrados, número de habitaciones, número de baños, tasas incluidas, datos adicionales (fotos, información sobre calefacción, garaje, ascensor, piscina, terraza, etc.).
	También se incluye si hay fianza.

	\item \textbf{RI-4} Historial

	Historial de anuncios de cada usuario del sistema.

	Identificador de usuario, lista de alquileres (identificadores)
	% Pagos que tiene pendientes

	\item \textbf{RI-5} Vehículos para alquilar

	Descripción de los vehículos disponibles para que otros usuarios los alquilen.

	Identificador del alquiler, identificador del dueño, identificador del vehículo, tipo de alquiler (total o parcial), datos adicionales.
	\begin{itemize}
		\item \textbf{RI-5.1} Alquiler total

		Alquiler del vehículo completo por un periodo de tiempo.

		Fianza, fechas de inicio y fin del alquiler, precio, lugar de recogida y devolución del vehículo, posibilidad de hacer subtrayectos.

		\item \textbf{RI-5.2} Alquiler plazas para viaje

		En este caso se alquilarían plazas sueltas para un viaje concreto.

		Fecha de inicio del viaje, fecha estimada de llegada (derivado del resto), lugar inicial y final del viaje, posibles paradas intermedias (para recoger otros pasajeros), número de plazas ofertadas y disponibles, precio por plaza, tamaño máximo del equipaje (pequeño, mediano o grande) y usuarios que ya han reservado alguna plaza en subtrayectos.
	\end{itemize}

	\item \textbf{RI-6} Viviendas para alquilar

	Viviendas actualmente disponibles para alquiler total o parcial.

	Identificador del anuncio, identificador de la vivienda, tipo de alquiler (total o por habitaciones).

	\begin{itemize}
		\item \textbf{RI-6.1} Alquiler total

		Alquiler de la vivienda completa por un periodo de tiempo

		Fecha inicio del alquiler, fecha fin del alquiler, precio por día alquilado.

		\item \textbf{RI-6.2} Alquiler por habitaciones

		Alquiler de habitaciones sueltas.

		Fecha inicio del alquiler, fecha fin del alquiler, precio por día alquilado, número de habitaciones para alquilar.
	\end{itemize}

	\item \textbf{RI-7} Normas
	\item \textbf{RI-8} Proveedores de seguros
	\item \textbf{RI-9} Contacto con la empresa
\end{itemize}

\subsection{Requisitos funcionales}
\begin{itemize}
	\item \textbf{RF-1} Gestión de usuarios
	\begin{itemize}
		\item \textbf{RF-1.1} Alta de usuario: Creación de usuarios nuevos en nuestra base de datos.

		\item \textbf{RF-1.2} Consulta de usuario: Ver las características del usuario.
		\begin{itemize}
			\item \textbf{RF-1.2.1} Datos personales: Información básica de contacto e identificación del usuario.
			\item \textbf{RF-1.2.2} Opiniones recibidas: Grado de satisfacción que tienen el resto de usuarios que han recibido servicios del mismo.
			\item \textbf{RF-1.2.3} Alquileres efectuados: Anuncios ofertados en la plataforma y que han sido contratados por algún usuario.
		\end{itemize}
		\item \textbf{RF-1.3} Modificar usuario
		\begin{itemize}
			\item \textbf{RF-1.3.1} Modificar datos personales: Modificar información básica de contacto e identificación del usuario.
			\begin{itemize}
				\item \textbf{RF-1.3.1.1} Verificar e-mail: Comprobación de haber recibido el e-mail de verificación.
				\item \textbf{RF-1.3.1.2} Verificar teléfono: Comprobación del teléfono mediante un mensaje de texto.
			\end{itemize}
			\item \textbf{RF-1.3.2} Añadir vehículo: Añadir vehículo a la lista de vehículos del usuario disponibles para alquilar.
			\item \textbf{RF-1.3.3} Añadir vivienda: Añadir vivienda a la lista de viviendas del usuario disponibles para alquilar.
			\item \textbf{RF-1.3.4} Modificar status: Modificar rango dentro de la plataforma en función de la experiencia del usuario y de las opiniones recibidas.
			\item \textbf{RF-1.3.5} Añadir alquiler efectuado: Guardar en la base de datos los datos referentes a dicho alquiler, eliminándolo de la lista de anuncios disponibles.
		\end{itemize}
		\item \textbf{RF-1.4} Gestión de pagos
		\begin{itemize}
			\item \textbf{RF-1.4.1} Consultar saldo: Ver el saldo virtual que tenemos dentro de la plataforma.
			\item \textbf{RF-1.4.2} Modificar saldo: Modificar el saldo indicado, se controla a base de privilegios, solamente puede ser modificado por administradores y por la realización de un alquiler.
			\item \textbf{RF-1.4.3} Pago a cuenta: Transferir el dinero virtual a una cuenta bancaria asociada al usuario.
		\end{itemize}
		\item \textbf{RF-1.5} Denunciar usuario: Advertir incidencia dentro de un alquiler. Dicha incidencia puede deberse al incumplimiento de la normativa o un mal uso del servicio.
		\item \textbf{RF-1.6} Eliminar usuario
	\end{itemize}

	\item \textbf{RF-2} Gestión de anuncios
	\begin{itemize}
		\item \textbf{RF-2.1} Añadir anuncio: Añadir alquileres a la lista de alquileres disponibles.
		\begin{itemize}
			\item \textbf{RF-2.1.1} Seleccionar vivienda: Selección, dentro de las viviendas disponibles para el usuario, de la vivienda que se desea poner en alquiler.
			\begin{itemize}
				\item \textbf{RF-2.1.1.1} Seleccionar habitación;  Selección de la habitación (dentro de una vivienda seleccionada para alquilar) que se desea poner en alquiler.
			\end{itemize}
			\item \textbf{RF-2.1.2} Seleccionar vehículo: Selección, dentro de los vehículos disponibles para el usuario, aquel que se desea poner en alquiler.
			\begin{itemize}
				\item \textbf{RF-2.1.2.1} Establecer número de plazas para compartir dentro de un trayecto.
			\end{itemize}
		\end{itemize}
		\item \textbf{RF-2.2} Consultar anuncio: Consulta de los detalles de un anuncio de alquiler.
		\item \textbf{RF-2.3} Modificar anuncio
		\begin{itemize}
			\item \textbf{RF-2.3.1} Modificar información anuncio:  Alteración de algún detalle del anuncio publicado.
			\item \textbf{RF-2.3.2} Confirmar alquiler: Aceptación, por parte del dueño del anuncio, de la reserva efectuada por el usuario que desea el servicio.
			\item \textbf{RF-2.3.3} Denegar alquiler: Rechazar la solicitud efectuada por parte del usuario que desea el servicio.
		\end{itemize}
		\item \textbf{RF-2.4} Gestión de mensajes
		\begin{itemize}
			\item \textbf{RF-2.4.1} Enviar mensaje: Publicar, en el tablón del anuncio, un mensaje para pedir información extra.
			\item \textbf{RF-2.4.2} Validar mensaje: Comprobar, por parte de los administradores del software, que el contenido del mensaje cumple con la normativa.
		\end{itemize}
		\item \textbf{RF-2.5} Dejar una calificación: Valorar el servicio recibido.
		\item \textbf{RF-2.6} Gestión de incidencias: 
		\begin{itemize}
			\item \textbf{RF-2.6.1} Notificar una incidencia: Avisar del incumplimiento de la normativa por parte de algún usuario.
			\item \textbf{RF-2.6.2} Consultar incidencias por parte del administrador.
			\item \textbf{RF-2.6.2} Resolución de incidencia: Resolución de alguna incidencia por parte del equipo administrador de la aplicación.	
		\end{itemize}
		\item \textbf{RF-2.7} Retirar alquiler de lista de ofertados
	\end{itemize}

	\item \textbf{RF-3} Gestión de administradores
	\begin{itemize}
		\item \textbf{RF-3.1} Gestión de privilegios: Asignar o retirar privilegios a los usuarios.
	\end{itemize}
\end{itemize}

\subsection{Requisitos no funcionales}
\begin{itemize}
	\item \textbf{RNF-1} Mantener coherencia de los datos
	\item \textbf{RNF-2} Privacidad de los datos personales de los usuarios
	\item \textbf{RNF-3} Interfaz intuitiva para alquilar	
\end{itemize}

\section{Glosario de términos}
\begin{itemize}
	\item Calificación: Puntuación numérica entre $0$ y $5$ (incluidos), donde $0$ es la menor calificación y $5$ la mayor.
	\item Economía colaborativa: Interacción entre dos o más sujetos, a través de medios digitalizados o no, que satisface una necesidad real o potencial, a una o más personas.	
	\item Saldo virtual: Dinero ficticio asociado a cada usuario que puede obtener al beneficiarse de alquileres (también podemos canjearlo por alquileres) y puede transferirse a una cuenta bancaria.
	\item Subtrayecto: Trayecto parcial del viaje, que empieza y/o acaba en paradas intermedias.
	\item Status: Nivel del usuario dentro de la plataforma, que se modifica en función de la experiencia y las opiniones recibidas de otros usuarios.
\end{itemize}


%%%%%%%%%%%%%%%%%%%%%%%%%%%%%%%%%%%%%%%%%%%%%%%%%%%%%%%%%%%%%%%%%%%%%%%%%%%%%%%%%%

% % % % % % % % % % % % % % % % % % % % % % % % % % % % % % % % %
%					 Bibliografía
% % % % % % % % % % % % % % % % % % % % % % % % % % % % % % % % %

% Para hacer citas: \cite{<referencia>}
% donde <referencia> es lo primero que aparece en el fichero .bib:
% @tipo{<referencia>, ...}

% Si no se utiliza una cita hay que especificarlo con \nocite{<referencia>}


% El orden de referencias es determinado por el orden de aparición de \cite y \nocite
\bibliographystyle{unsrt}

\bibliography{./Bibliografia/preambulo,./Bibliografia/referencias}


\end{document}
